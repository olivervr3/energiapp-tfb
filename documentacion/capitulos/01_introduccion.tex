\chapter{Introducción}
\label{ch:introduccion}

\section{Presentación del problema}

En la actualidad, el consumo energético residencial representa apr\subsection{Herramientas y tecnologías avanzadas}

El stack tecnológico de EnergiApp v1.0 se ha seleccionado estratégicamente considerando factores como madurez tecnológica, escalabilidad, performance, disponibilidad de documentación extensiva, robustez de la comunidad de desarrolladores, capacidades de integración con inteligencia artificial, y alineación perfecta con los objetivos de innovación del proyecto.

La arquitectura de backend está basada en Node.js 18+ con Express.js, incorporando middleware de seguridad avanzado, base de datos simulada in-memory optimizada para desarrollo y demo, Helmet para protección adicional, configuración CORS específica, y un sistema de logging profesional que permite el monitoreo exhaustivo del rendimiento del sistema.

El frontend de próxima generación utiliza React 18 con Hooks avanzados, implementando CSS Grid y Flexbox para lograr un diseño responsive profesional. Las visualizaciones interactivas se desarrollan con Chart.js para representación gráfica de datos ML, mientras que Axios proporciona un cliente HTTP optimizado para comunicación eficiente con el backend, todo estructurado en una arquitectura de componentes escalable.

El componente de inteligencia artificial y machine learning incorpora algoritmos de predicción personalizados basados en comportamientos de dispositivos reales, un motor de recomendaciones que genera sugerencias ejecutables, sistema de automatización temporal avanzado, y capacidades de análisis profundo de patrones de consumo energético para optimización continua.

Las herramientas de desarrollo profesional incluyen Git para control de versiones distribuido, VS Code con extensiones especializadas para desarrollo fullstack, capacidades de debugging avanzado para identificación rápida de problemas, hot reload para desarrollo ágil con feedback inmediato, y testing automatizado para garantizar la calidad del código.

En el contexto actual de crisis climática, el sector residencial representa el 25\% del consumo total de energía a nivel mundial \cite{iea2023}. La Unión Europea ha establecido objetivos ambiciosos para 2030, incluyendo una reducción del 55\% de las emisiones con respecto a los niveles de 1990 \cite{european_green_deal}.

Los hogares modernos albergan múltiples dispositivos electrónicos cuyo consumo energético conjunto representa una parte sustancial del gasto familiar. Sin embargo, la mayoría de usuarios domésticos carecen de herramientas avanzadas para optimizar automáticamente su consumo energético.

Los sistemas tradicionales de medición se limitan a datos agregados mensuales, resultando insuficientes para la identificación de patrones granulares de consumo o la optimización automatizada. Esta limitación constituye una barrera para la adopción de hábitos sostenibles.

\section{Justificación del proyecto}

\subsection{Relevancia social y ambiental en la era digital}

El desarrollo de EnergiApp v1.0 como herramienta de gestión inteligente y automatizada del consumo energético doméstico se ha convertido en una prioridad estratégica tanto a nivel político como tecnológico y social. Los beneficios transformadores de este tipo de soluciones de inteligencia artificial aplicada son múltiples y exponencialmente escalables.

La reducción automática de emisiones constituye uno de los beneficios más significativos, ya que una gestión inteligente basada en IA del consumo energético doméstico puede contribuir de manera medible y automatizada a la reducción de la huella de carbono de los hogares, generando un impacto directo y cuantificable en los objetivos climáticos globales establecidos por acuerdos internacionales.

La optimización económica inteligente representa otro pilar fundamental del proyecto. La identificación automática de patrones de consumo ineficientes, combinada con la ejecución de recomendaciones en tiempo real, permite a los usuarios reducir significativamente sus gastos en electricidad mediante decisiones optimizadas por algoritmos de machine learning que operan de forma continua y adaptativa.

La democratización de la inteligencia artificial constituye un aspecto revolucionario de EnergiApp v1.0, ya que hace accesible la potencia de algoritmos de predicción avanzados y automatización inteligente a usuarios domésticos sin conocimientos técnicos especializados, eliminando barreras tecnológicas tradicionales y facilitando la adopción masiva de tecnologías de optimización energética.

El potencial de escalabilidad hacia comunidades inteligentes está integrado en la arquitectura fundamental del sistema, proporcionando la base tecnológica necesaria para la expansión hacia redes de gestión energética comunitaria y el desarrollo de infraestructuras de ciudades inteligentes que optimicen el consumo energético a nivel metropolitano.

El impacto educativo y de concienciación se materializa a través de herramientas de visualización avanzada, predicciones ML interpretables y sistemas de automatización inteligente que fomentan una comprensión profunda y práctica del impacto ambiental del consumo energético, transformando el comportamiento del usuario a través de feedback educativo continuo.

\subsection{Oportunidad tecnológica e innovación disruptiva}

El avance exponencial de las tecnologías de Internet of Things (IoT), inteligencia artificial, machine learning aplicado, y desarrollo web de nueva generación ha creado un escenario tecnológico propicio para el desarrollo de soluciones verdaderamente disruptivas en el ámbito de la gestión energética doméstica inteligente. La convergencia sinérgica de estos factores tecnológicos revolucionarios permite múltiples capacidades avanzadas.

La recopilación y procesamiento de datos granulares en tiempo real sobre el consumo energético de dispositivos individuales con precisión de segundos representa una capacidad fundamental que habilita el análisis detallado y la toma de decisiones optimizadas basada en información precisa y actualizada continuamente.

El procesamiento distribuido y análisis predictivo de grandes volúmenes de datos energéticos mediante algoritmos de machine learning optimizados permite la identificación de patrones complejos y la generación de predicciones precisas que superan significativamente las capacidades de análisis tradicionales.

La aplicación de técnicas avanzadas de inteligencia artificial para predicción temporal, detección automática de anomalías y generación de recomendaciones ejecutables transforma la gestión energética de un proceso reactivo a uno proactivo y adaptativo que anticipa necesidades y optimiza recursos automáticamente.

El desarrollo de interfaces web de próxima generación con capacidades responsive, visualizaciones interactivas y experiencia de usuario excepcional democratiza el acceso a tecnologías avanzadas, haciendo que herramientas sofisticadas sean accesibles para usuarios con diferentes niveles de competencia tecnológica.

La implementación de sistemas de automatización inteligente con capacidades de control temporal y optimización proactiva de dispositivos reales permite la ejecución automática de estrategias de eficiencia energética sin requerir intervención manual constante del usuario.

La integración de algoritmos de optimización energética que ejecutan acciones automáticas sobre dispositivos domésticos para maximizar eficiencia representa el siguiente nivel evolutivo en gestión energética doméstica, donde la inteligencia artificial opera de forma autónoma para lograr objetivos de sostenibilidad y ahorro económico.

\subsection{Alineación con los Objetivos de Desarrollo Sostenible}

Este proyecto se alinea directamente con varios de los Objetivos de Desarrollo Sostenible (ODS) establecidos por las Naciones Unidas:

\begin{description}
    \item[ODS 7 - Energía asequible y no contaminante:] La plataforma contribuye a garantizar el acceso a una energía asequible, segura, sostenible y moderna para todos, promoviendo la eficiencia energética y el uso responsable de los recursos.
    
    \item[ODS 11 - Ciudades y comunidades sostenibles:] Al facilitar la gestión eficiente del consumo energético en los hogares, el proyecto contribuye a hacer que las ciudades sean más inclusivas, seguras, resilientes y sostenibles.
    
    \item[ODS 12 - Producción y consumo responsables:] La herramienta promueve modalidades de consumo sostenibles mediante la concienciación y la facilitación de decisiones informadas sobre el uso de la energía.
    
    \item[ODS 13 - Acción por el clima:] La reducción del consumo energético doméstico contribuye directamente a la lucha contra el cambio climático y sus efectos.
\end{description}

\section{Objetivos}

\subsection{Objetivo general}

Desarrollar e implementar EnergiApp v1.0, una plataforma web inteligente de próxima generación que permita a los usuarios visualizar, predecir y optimizar automáticamente su consumo energético doméstico a través del análisis de datos avanzado, algoritmos de machine learning aplicado, automatización inteligente de dispositivos, y simulación representativa de ecosistemas IoT, con el objetivo estratégico de transformar la sostenibilidad energética en el hogar mediante inteligencia artificial accesible y ejecutable.

\subsection{Objetivos específicos avanzados}

\begin{enumerate}
    \item \textbf{Desarrollo de arquitectura ML avanzada:} Implementar algoritmos de machine learning consistentes y realistas que generen predicciones energéticas de 1-7 días basadas en patrones reales de dispositivos domésticos, eliminando la aleatoriedad tradicional y proporcionando datos útiles para toma de decisiones optimizadas.
    
    \item \textbf{Sistema de recomendaciones ejecutables:} Crear un motor de inteligencia artificial que no solo identifique oportunidades de optimización energética, sino que ejecute automáticamente acciones sobre dispositivos reales del usuario (optimización standby, control climático, programación temporal) con validación previa y feedback inmediato.
    
    \item \textbf{Automatización inteligente temporal:} Implementar capacidades de programación automática de electrodomésticos que ejecuten control temporal real (apagar ahora → encender programado) para aprovechar tarifas valle y optimizar costos energéticos con demostración funcional en tiempo real.
    
    \item \textbf{Dashboard ejecutivo en tiempo real:} Desarrollar un panel de control profesional con métricas KPI del sistema, análisis comparativo, visualizaciones Chart.js interactivas y notificaciones inteligentes que proporcionen feedback visual inmediato sobre acciones de optimización realizadas.
    
    \item \textbf{Arquitectura backend robusta y escalable:} Implementar una API RESTful avanzada con más de 30 endpoints, sistema de autenticación con roles diferenciados (admin/usuario), middleware de seguridad, y gestión completa multi-usuario con operaciones CRUD exhaustivas.
    
    \item \textbf{Interfaz responsive de excelencia:} Crear una aplicación frontend con React 18 que incluya navegación horizontal optimizada, diseño mobile-first, modales informativos detallados, animaciones fluidas y experiencia de usuario excepcional en todos los dispositivos.
    
    \item \textbf{Sistema administrativo completo:} Desarrollar un panel de administración integral que permita gestión avanzada de usuarios (crear, activar, desactivar, eliminar), control global de dispositivos, logs del sistema en tiempo real, y generación de reportes energéticos profesionales.
    
    \item \textbf{Validación y optimización de rendimiento:} Realizar testing exhaustivo que garantice tiempos de respuesta API inferiores a 100ms, compilación frontend exitosa, precisión predictiva superior al 90\%, y funcionalidad completa de todas las características implementadas.
    
    \item \textbf{Documentación académica y técnica profesional:} Crear documentación exhaustiva que incluya manual de usuario de 740+ líneas, guía de demostración de 15 minutos, documentación técnica LaTeX actualizada, y materiales de presentación académica de nivel profesional.
    
    \item \textbf{Innovación en experiencia de usuario:} Implementar funcionalidades revolucionarias como tarjetas predictivas dinámicas con información meteorológica integrada, modales de información detallada sobre tecnologías sostenibles (paneles solares, ROI, subvenciones), y sistema de notificaciones con cálculo de ahorros reales.
\end{enumerate}

\section{Metodología}

\subsection{Enfoque metodológico}

Para el desarrollo de este proyecto se ha adoptado una metodología ágil basada en Scrum, adaptada a las características de un proyecto académico individual. Esta metodología permite un desarrollo incremental e iterativo, facilitando la adaptación a los cambios y la mejora continua del producto.

\subsection{Fases del proyecto}

El proyecto se ha estructurado en las siguientes fases principales:

\begin{enumerate}
    \item \textbf{Fase de investigación y análisis (Semanas 1-3):} Esta fase inicial comprende la revisión exhaustiva de literatura científica sobre IoT, gestión energética y machine learning, el análisis detallado de datasets existentes de consumo energético, el estudio profundo de tecnologías y herramientas disponibles en el mercado, y la definición precisa de requisitos funcionales y no funcionales que guiarán todo el desarrollo.
    
    \item \textbf{Fase de diseño (Semanas 4-5):} Durante esta etapa se desarrolla el diseño completo de la arquitectura del sistema, incluyendo la definición del modelo de datos optimizado, el diseño de la interfaz de usuario centrada en la experiencia del usuario, y la especificación detallada de la API RESTful que conectará todos los componentes del sistema.
    
    \item \textbf{Fase de desarrollo backend (Semanas 6-8):} Esta fase se centra en la implementación de la API RESTful robusta y escalable, el desarrollo del sistema de gestión de usuarios con diferentes roles y permisos, la implementación del sistema de simulación de datos IoT realista, y el desarrollo de los endpoints especializados para gestión de consumo y generación de predicciones.
    
    \item \textbf{Fase de desarrollo de modelos ML (Semanas 9-10):} Durante este período se implementan algoritmos de predicción avanzados, se realiza el entrenamiento y validación rigurosa de modelos usando datasets reales, se ejecuta la integración completa con el backend, y se lleva a cabo la optimización de rendimiento para garantizar respuestas en tiempo real.
    
    \item \textbf{Fase de desarrollo frontend (Semanas 11-13):} Esta etapa incluye la implementación de componentes React modernos y reutilizables, el desarrollo de visualizaciones interactivas que faciliten la comprensión de datos complejos, la integración completa con la API backend para funcionalidad en tiempo real, y la implementación de todas las funcionalidades de usuario con focus en usabilidad.
    
    \item \textbf{Fase de testing y validación (Semanas 14-15):} Esta fase crítica abarca pruebas unitarias e integración exhaustivas, validación rigurosa de modelos predictivos con métricas de precisión, pruebas de usabilidad con usuarios reales para garantizar experiencia óptima, y optimización de rendimiento en todos los componentes del sistema.
    
    \item \textbf{Fase de documentación (Semanas 16-18):} La fase final comprende la redacción completa de la memoria del TFG con estándares académicos, la creación de manuales de usuario detallados y accesibles, la documentación técnica exhaustiva del código para mantenimiento futuro, y la preparación de presentaciones académicas profesionales.
\end{enumerate}

\subsection{Herramientas y tecnologías}

Las tecnologías seleccionadas para el desarrollo del proyecto se han elegido considerando factores como la madurez tecnológica, la disponibilidad de documentación, la comunidad de desarrolladores y la alineación con los objetivos del proyecto.

Para el desarrollo del backend se ha optado por Node.js como runtime de JavaScript, Express.js como framework web por su flexibilidad y rendimiento, PostgreSQL como sistema de gestión de base de datos relacional robusto, y Sequelize ORM para facilitar las operaciones de base de datos y mantener la integridad referencial.

El frontend utiliza React como biblioteca principal para el desarrollo de interfaces de usuario interactivas, TypeScript para añadir tipado estático y mejorar la robustez del código, Material-UI como sistema de diseño para garantizar consistencia visual, y Chart.js para crear visualizaciones de datos atractivas e informativas.

Los componentes de machine learning están implementados en Python aprovechando su ecosistema maduro para ciencia de datos, utilizando scikit-learn para algoritmos de aprendizaje automático, pandas para manipulación y análisis de datos, y NumPy para operaciones numéricas eficientes y cálculos matriciales.

Las herramientas de desarrollo incluyen Git para control de versiones distribuido que facilita la colaboración, VS Code como entorno de desarrollo integrado con extensiones especializadas, Docker para containerización y despliegue consistente, y Jest para testing automatizado que garantiza la calidad del código.

La documentación se gestiona utilizando LaTeX para la generación de documentos académicos de alta calidad, y Swagger/OpenAPI para la documentación automática de la API RESTful, facilitando la comprensión y uso de los endpoints por parte de desarrolladores y usuarios técnicos.

\section{Estructura del documento}

Este documento se organiza en los siguientes capítulos:

\begin{description}
    \item[Capítulo 2 - Marco teórico:] Presenta los fundamentos teóricos y el estado del arte en las áreas de IoT, análisis de datos energéticos, machine learning y desarrollo web.
    
    \item[Capítulo 3 - Análisis del problema y diseño de la solución:] Detalla el análisis de requisitos, el diseño de la arquitectura del sistema y las decisiones técnicas adoptadas.
    
    \item[Capítulo 4 - Desarrollo técnico:] Describe la implementación de los diferentes componentes del sistema, incluyendo backend, frontend y modelos de machine learning.
    
    \item[Capítulo 5 - Resultados y validación:] Presenta los resultados obtenidos, las pruebas realizadas y la validación del sistema desarrollado.
    
    \item[Capítulo 6 - Conclusiones y trabajo futuro:] Resume las conclusiones del proyecto y propone líneas de trabajo futuro.
\end{description}

Adicionalmente, se incluyen varios apéndices con información complementaria sobre el código desarrollado, manuales de usuario e instalación, y detalles sobre los datasets utilizados.

