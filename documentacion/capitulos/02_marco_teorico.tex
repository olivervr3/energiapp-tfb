\chapter{Marco Teórico}
\label{ch:marco_teorico}

\section{IoT y gestión energética doméstica}

\subsection{Arquitectura IoT para sistemas energéticos}

Los sistemas IoT energéticos implementan una arquitectura de cuatro capas especializadas:

\begin{table}[H]
\centering
\caption{Arquitectura de capas en sistemas IoT energéticos}
\begin{tabular}{|l|l|l|}
\hline
\textbf{Capa} & \textbf{Componentes} & \textbf{Función Principal} \\
\hline
Percepción & Medidores inteligentes, sensores & Captura de datos energéticos \\
\hline
Conectividad & WiFi, Zigbee 3.0, LoRaWAN & Transmisión de datos \\
\hline
Procesamiento & Edge/Cloud computing & Análisis y algoritmos ML \\
\hline
Aplicación & Interfaces de usuario & Visualización y control \\
\hline
\end{tabular}
\label{tab:arquitectura_iot}
\end{table}

\textbf{Impacto cuantificado:} Estudios empíricos demuestran reducción del consumo energético entre 10-23\% mediante sistemas IoT \cite{iea2022digitalization}, aunque la adopción real permanece bajo 15\% en países desarrollados.

\subsection{Evolución tecnológica}

\textbf{Primera generación (2008-2015):} Medidores básicos unidireccionales, granularidad 15-60 minutos.

\textbf{Segunda generación (2016-2021):} Medición sub-métrica bidireccional, técnicas NILM, granularidad 1-5 minutos.

\textbf{Tercera generación (2022-presente):} Machine learning distribuido, medición en tiempo real (<1s), identificación automática de dispositivos.

\section{Machine learning para predicción energética}

\subsection{Características específicas de datos energéticos}

Los datos energéticos domésticos presentan desafíos únicos para el análisis computacional:

\begin{itemize}
    \item \textbf{Estacionalidad múltiple:} Patrones diarios, semanales, mensuales y anuales superpuestos
    \item \textbf{Dependencia contextual:} Influencia de factores meteorológicos y sociales
    \item \textbf{Heteroscedasticidad temporal:} Varianza no constante según períodos del día
    \item \textbf{Datos anómalos:} Fallos de conectividad y eventos excepcionales
\end{itemize}

\subsection{Algoritmos de machine learning aplicados}

\textbf{Random Forest/Gradient Boosting:} Robustez ante outliers, captura de relaciones no lineales. Limitación: interpretabilidad reducida.

\textbf{Support Vector Regression:} Excelente rendimiento en datasets medianos. Limitación: escalabilidad.

\textbf{Redes neuronales LSTM/GRU:} Captura de dependencias temporales a largo plazo. Limitación: requisitos de datos masivos y riesgo de overfitting.

\subsection{Metodología de evaluación especializada}

\textbf{Validación temporal:} Forward-chaining respetando cronología de datos.

\textbf{Métricas específicas:} RMSE, MAPE, Peak Hour Error Rate para períodos críticos.

\textbf{Análisis de incertidumbre:} Estimaciones de confianza para decisiones de optimización.

Las variantes regularizadas Ridge y Lasso introducen términos de penalización que previenen el sobreajuste, especialmente crucial cuando el número de variables predictoras es alto relative al número de observaciones. Ridge regression utiliza penalización L2 que reduce la magnitud de los coeficientes, mientras que Lasso emplea penalización L1 que puede llevar algunos coeficientes exactamente a cero, proporcionando selección automática de características.

Support Vector Regression (SVR) extiende las capacidades predictivas al espacio no lineal mediante el uso de kernels, siendo particularmente efectivo para capturar relaciones complejas entre variables que los modelos lineales no pueden representar adecuadamente. Su robustez frente a outliers lo hace especialmente valioso en datos energéticos reales que frecuentemente contienen mediciones anómalas.

\subsubsection{Modelos de ensemble}

Los métodos de ensemble representan una evolución significativa en el machine learning aplicado a predicción energética, combinando múltiples modelos para mejorar sustancialmente la precisión y robustez de las predicciones compared to single-model approaches.

Random Forest implementa una estrategia de bagging que combina múltiples árboles de decisión entrenados en diferentes subconjuntos de datos y características, reduciendo efectivamente la varianza del modelo final. Esta técnica es particularmente valiosa en datos energéticos debido a su capacidad para manejar relaciones no lineales complejas y su robustez inherente frente a outliers y datos faltantes.

Gradient Boosting adopta un enfoque secuencial donde cada modelo nuevo se construye específicamente para corregir los errores de predicción cometidos por los modelos anteriores. Esta metodología iterativa permite capturar patrones sutiles en los datos energéticos que modelos individuales podrían pasar por alto, resultando en predicciones más precisas.

XGBoost y LightGBM representan implementaciones optimizadas de gradient boosting que incorporan mejoras algorítmicas y de rendimiento significativas. XGBoost utiliza regularización avanzada y optimizaciones de memoria que lo hacen especialmente efectivo para datasets de gran escala típicos en aplicaciones IoT. LightGBM emplea leaf-wise tree growth en lugar del level-wise tradicional, reduciendo significativamente el tiempo de entrenamiento mientras mantiene alta precisión predictiva.

\subsubsection{Redes neuronales}

Las redes neuronales han revolucionado el análisis de series temporales energéticas al proporcionar capacidades de modelado no lineal que superan significativamente los enfoques tradicionales, especialmente en la captura de dependencias complejas a largo plazo características de los patrones de consumo energético.

Las redes LSTM (Long Short-Term Memory) representan un avance fundamental en el procesamiento de secuencias temporales, incorporando mecanismos de memoria selectiva que permiten retener información relevante a través de intervalos temporales extensos mientras olvidan información irrelevante. Esta capacidad es crucial para modelar patrones energéticos que pueden depender de eventos ocurridos días o semanas anteriores, como cambios estacionales o hábitos de usuario establecidos \cite{shi2018deep}.

Las unidades GRU (Gated Recurrent Units) constituyen una variante simplificada pero efectiva de LSTM que reduce la complejidad computacional mediante la combinación de las puertas de olvido y entrada en una sola puerta de actualización. Esta simplificación resulta en menor coste computacional y tiempo de entrenamiento, manteniendo capacidades predictivas comparables, lo que las hace especialmente atractivas para aplicaciones en tiempo real con limitaciones de recursos.

La arquitectura Transformer, inicialmente desarrollada para procesamiento de lenguaje natural, ha emergido como una alternativa prometedora para series temporales energéticas. Su mecanismo de atención permite al modelo identificar y ponderar automáticamente las relaciones más relevantes entre diferentes momentos temporales, independientemente de su distancia en la secuencia, superando las limitaciones de dependencia secuencial de las redes recurrentes tradicionales.

\section{Desarrollo web moderno}

\subsection{Stack tecnológico seleccionado}

El proyecto utiliza un stack tecnológico moderno optimizado para aplicaciones IoT con requisitos de tiempo real:

\begin{table}[H]
\centering
\caption{Stack tecnológico implementado}
\begin{tabular}{|l|l|l|}
\hline
\textbf{Capa} & \textbf{Tecnología} & \textbf{Justificación} \\
\hline
Frontend & React + TypeScript & SPA responsivo, tipado estático \\
\hline
Backend & Node.js + Express & Arquitectura orientada a eventos \\
\hline
Base de datos & SQLite & Simplicidad, sin dependencias \\
\hline
ML & Python + scikit-learn & Ecosistema maduro para ML \\
\hline
\end{tabular}
\label{tab:stack_tecnologico}
\end{table}

\section{Estado del arte en plataformas de gestión energética}

\subsection{Soluciones comerciales existentes}

\begin{table}[H]
\centering
\caption{Comparativa de plataformas comerciales}
\begin{tabular}{|l|l|l|l|}
\hline
\textbf{Plataforma} & \textbf{Enfoque} & \textbf{Fortalezas} & \textbf{Limitaciones} \\
\hline
Google Nest & Climatización & Integración ecosistema & Dominio limitado \\
\hline
Schneider EcoStruxure & Empresarial & Robustez industrial & Complejidad/coste \\
\hline
Sense Energy Monitor & Monitoreo residencial & Instalación simple & Precisión variable \\
\hline
\end{tabular}
\label{tab:plataformas_comerciales}
\end{table}

\subsection{Brechas identificadas}

Las soluciones actuales presentan limitaciones significativas:

\begin{table}[H]
\centering
\caption{Brechas identificadas en soluciones existentes}
\begin{tabular}{|l|l|}
\hline
\textbf{Brecha} & \textbf{Descripción} \\
\hline
Accesibilidad & Hardware costoso, conocimientos técnicos avanzados \\
\hline
Interoperabilidad & Falta de estándares comunes entre dispositivos \\
\hline
Privacidad & Manejo inadecuado de datos personales \\
\hline
Adaptabilidad & Poco contexto cultural y regulatorio local \\
\hline
Educación & Ausencia de herramientas educativas integradas \\
\hline
\end{tabular}
\label{tab:brechas_identificadas}
\end{table}

Este proyecto busca abordar estas brechas mediante una solución open-source, accesible y educativa.

