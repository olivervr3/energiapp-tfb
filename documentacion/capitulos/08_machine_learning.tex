\chapter{Metodologías Avanzadas de Machine Learning para Predicción Energética}
\label{ch:machine_learning}

\section{Arquitectura de Modelos de Machine Learning}

\subsection{Diseño del ensemble de predictores especializados}

La complejidad inherente de los patrones de consumo energético doméstico requiere una aproximación multi-modelo que capture tanto las características generales del consumo agregado como los patrones específicos de cada dispositivo individual. El sistema implementado utiliza un ensemble de predictores especializados con arquitectura jerárquica.

\subsubsection{Arquitectura general del sistema}

\begin{figure}[H]
\centering
\begin{tikzpicture}[
    node distance=1.5cm,
    every node/.style={rectangle, draw, text centered, minimum height=1cm},
    arrow/.style={-stealth, thick}
]
    % Entrada de datos
    \node (input) [fill=blue!20] {Dataset UK-DALE\\432,000 muestras};
    
    % Preprocesamiento
    \node (preprocessing) [below of=input, fill=green!20] {Pipeline Preprocesamiento\\Feature Engineering};
    
    % Modelos especializados
    \node (aggregate) [below left=2cm and -1cm of preprocessing, fill=yellow!20] {Predictor\\Agregado};
    \node (appliance1) [below=2cm of preprocessing, fill=orange!20] {Predictores\\Dispositivos};
    \node (appliance2) [below right=2cm and -1cm of preprocessing, fill=orange!20] {Modelos\\Individuales};
    
    % Ensemble
    \node (ensemble) [below=2cm of appliance1, fill=red!20] {Ensemble\\Meta-learner};
    
    % Salida
    \node (output) [below of=ensemble, fill=purple!20] {Predicción Final\\Consumo + Dispositivos};
    
    % Flechas
    \draw [arrow] (input) -- (preprocessing);
    \draw [arrow] (preprocessing) -- (aggregate);
    \draw [arrow] (preprocessing) -- (appliance1);
    \draw [arrow] (preprocessing) -- (appliance2);
    \draw [arrow] (aggregate) -- (ensemble);
    \draw [arrow] (appliance1) -- (ensemble);
    \draw [arrow] (appliance2) -- (ensemble);
    \draw [arrow] (ensemble) -- (output);
\end{tikzpicture}
\caption{Arquitectura del sistema de machine learning para predicción energética}
\label{fig:ml_architecture}
\end{figure}

\subsubsection{Especificaciones técnicas de los modelos}

\textbf{1. Predictor de Consumo Agregado}

El predictor de consumo agregado utiliza un modelo híbrido que combina análisis temporal profundo con características estacionales:

\subsubsection{Implementación del predictor agregado}

El predictor agregado representa el componente central del sistema de machine learning, diseñado específicamente para predecir el consumo energético total del hogar. Su implementación se basa en un ensemble de algoritmos de gradient boosting que combina XGBoost, LightGBM y Gradient Boosting clásico.

**Arquitectura del ensemble:** El sistema integra tres algoritmos complementarios donde XGBoost proporciona robustez y capacidad de generalización, LightGBM aporta eficiencia computacional y manejo optimizado de features categóricas, mientras que Gradient Boosting clásico ofrece estabilidad predictiva. Los pesos del ensemble se optimizaron mediante validación cruzada temporal para maximizar la precisión predictiva.

**Features temporales avanzadas:** La arquitectura incorpora features temporales que capturan múltiples patrones estacionales. Las features cíclicas utilizan transformaciones sinusoidales para representar la naturaleza circular del tiempo (hora del día, día de la semana, mes del año), mientras que las features de lag capturan dependencias temporales a diferentes horizontes temporales (6 segundos, 24 horas, 7 días).

**Estadísticas móviles:** El sistema implementa rolling statistics que calculan medias, desviaciones estándar y volatilidad sobre ventanas temporales deslizantes, proporcionando al modelo información sobre tendencias recientes y variabilidad del consumo energético.

**Metodología de entrenamiento:** El entrenamiento del ensemble utiliza validación cruzada temporal que respeta la naturaleza secuencial de los datos energéticos, evitando data leakage y proporcionando estimaciones realistas del rendimiento en producción.

\subsection{Preprocesamiento avanzado de features}
                df['trend_component'] = df['power']
                df['residual_component'] = 0
El sistema implementa un enfoque de descomposición estacional avanzada que identifica componentes temporales, tendencias y residuos para mejorar la calidad predictiva. La periodicidad semanal se captura mediante análisis de frecuencia que considera ciclos de 1008 períodos de medición.

La implementación del ensemble permite combinar las predicciones de múltiples modelos mediante pesos optimizados, mejorando la robustez y precisión predictiva del sistema.

\subsection{Predictores especializados por dispositivo}

Cada tipo de electrodoméstico requiere un enfoque predictivo específico debido a sus patrones de uso únicos. El sistema implementa predictores especializados que adaptan tanto la arquitectura del modelo como las features utilizadas según el tipo de dispositivo.

**Electrodomésticos cíclicos:** Para dispositivos como lavadoras y lavavajillas, se implementa detección de inicio y fin de ciclo mediante clasificación binaria seguida de regresión para la predicción de consumo durante el ciclo activo.

**Electrodomésticos continuos:** Dispositivos como frigoríficos requieren análisis de eficiencia energética y detección de patrones de comportamiento térmico mediante modelos de regresión continua con baseline adaptativo.

**Dispositivos de entretenimiento:** Para dispositivos como televisiones, se utilizan modelos de clasificación binaria que consideran patrones temporales y hábitos de uso específicos del usuario.

Los predictores especializados implementan diferentes estrategias según el comportamiento del dispositivo:

**Electrodomésticos cíclicos (lavadora, lavavajillas):** Utilizan un modelo híbrido de clasificación-regresión que primero determina si el dispositivo está en uso y luego predice el consumo específico durante el ciclo.

**Electrodomésticos continuos (frigorífico):** Emplean regresión continua con features de eficiencia térmica y análisis de tendencias de consumo a largo plazo.

**Dispositivos de entretenimiento (televisión):** Implementan clasificación binaria enfocada en patrones de uso temporal y preferencias de usuario.

**Dispositivos de iluminación:** Utilizan regresión multi-nivel que considera la luz natural disponible y patrones de ocupación inferidos.

\subsection{Técnicas avanzadas de optimización de hiperparámetros}

La optimización de hiperparámetros constituye un componente crítico que determina el rendimiento final de los modelos. Implementamos una estrategia multi-nivel que combina búsqueda bayesiana, optimización evolutiva y validación temporal específica para datos energéticos.

\subsection{Optimización de hiperparámetros}

La optimización de hiperparámetros utiliza búsqueda bayesiana mediante Optuna para encontrar la configuración óptima de cada modelo. Este enfoque es más eficiente que grid search tradicional, especialmente para espacios de hiperparámetros de alta dimensionalidad.

**Búsqueda bayesiana avanzada:** El sistema implementa Tree-structured Parzen Estimator (TPE) como sampler, que modela la distribución de parámetros prometedores basándose en trials anteriores. Esto permite convergencia más rápida hacia configuraciones óptimas comparado con búsqueda aleatoria.

**Métricas de optimización personalizadas:** Se implementan métricas específicas para datos energéticos, como energy-weighted MAE que penaliza más los errores durante períodos de alto consumo, reflejando la importancia práctica de predicciones precisas durante picos de demanda.

**Validación temporal:** La validación cruzada respeta la naturaleza temporal de los datos, utilizando TimeSeriesSplit para evitar data leakage y obtener estimaciones realistas del rendimiento del modelo en producción.

Esta metodología integral de Machine Learning asegura que los modelos de predicción energética alcancen la máxima precisión posible utilizando técnicas estado del arte adaptadas específicamente para el dominio de datos energéticos domésticos.
