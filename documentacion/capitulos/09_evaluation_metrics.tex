\chapter{Evaluación de Modelos y Métricas de Rendimiento}
\label{ch:evaluation_metrics}

\section{Framework de Evaluación Integral}

\subsection{Metodología de evaluación multi-dimensional}

La evaluación de modelos de predicción energética requiere un enfoque multifacético que capture tanto la precisión numérica como la utilidad práctica en aplicaciones reales de gestión energética. El framework implementado evalúa los modelos desde múltiples perspectivas: precisión estadística, estabilidad temporal, interpretabilidad física y aplicabilidad práctica.

\subsubsection{Arquitectura del sistema de evaluación}

\begin{figure}[H]
\centering
\begin{tikzpicture}[
    node distance=1.5cm,
    every node/.style={rectangle, draw, text centered, minimum height=0.8cm, minimum width=2cm},
    arrow/.style={-stealth, thick}
]
    % Entrada
    \node (predictions) [fill=blue!20] {Predicciones\\Modelo};
    \node (ground_truth) [right=2cm of predictions, fill=blue!20] {Ground Truth\\UK-DALE};
    
    % Métricas básicas
    \node (basic_metrics) [below=1cm of predictions, fill=green!20] {Métricas\\Básicas};
    \node (temporal_metrics) [right=1cm of basic_metrics, fill=green!20] {Métricas\\Temporales};
    \node (energy_metrics) [right=1cm of temporal_metrics, fill=green!20] {Métricas\\Energéticas};
    
    % Análisis avanzado
    \node (stability) [below=1cm of basic_metrics, fill=yellow!20] {Análisis\\Estabilidad};
    \node (interpretability) [right=1cm of stability, fill=yellow!20] {Interpretabilidad};
    \node (practical) [right=1cm of interpretability, fill=yellow!20] {Aplicabilidad\\Práctica};
    
    % Resultado final
    \node (final_score) [below=1.5cm of interpretability, fill=red!20] {Score Integral\\de Evaluación};
    
    % Flechas
    \draw [arrow] (predictions) -- (basic_metrics);
    \draw [arrow] (ground_truth) -- (basic_metrics);
    \draw [arrow] (predictions) -- (temporal_metrics);
    \draw [arrow] (ground_truth) -- (temporal_metrics);
    \draw [arrow] (predictions) -- (energy_metrics);
    \draw [arrow] (ground_truth) -- (energy_metrics);
    
    \draw [arrow] (basic_metrics) -- (stability);
    \draw [arrow] (temporal_metrics) -- (interpretability);
    \draw [arrow] (energy_metrics) -- (practical);
    
    \draw [arrow] (stability) -- (final_score);
    \draw [arrow] (interpretability) -- (final_score);
    \draw [arrow] (practical) -- (final_score);
\end{tikzpicture}
\caption{Arquitectura del framework de evaluación multi-dimensional}
\label{fig:evaluation_framework}
\end{figure}

\subsubsection{Implementación del evaluador integral}

El sistema de evaluación implementa una clase \texttt{EnergyModelEvaluator} que proporciona análisis multidimensional de rendimiento para modelos de predicción energética. La evaluación comprende:

\begin{itemize}
    \item \textbf{Métricas básicas:} MAE, RMSE, R², MAPE, correlaciones Pearson y Spearman
    \item \textbf{Análisis temporal:} Rendimiento por ventanas temporales, detección de drift
    \item \textbf{Métricas energéticas:} Precisión por niveles de consumo, detección de picos
    \item \textbf{Estabilidad del modelo:} Análisis de homocedasticidad, outliers, autocorrelación
    \item \textbf{Consistencia física:} Validación de no negatividad, gradientes razonables
    \item \textbf{Utilidad práctica:} Precisión para facturación, detección de anomalías
\end{itemize}

La implementación utiliza bibliotecas especializadas (scikit-learn, scipy, statsmodels) para cálculo robusto de métricas estadísticas y tests de significancia.

\section{Métricas Específicas para Aplicaciones Energéticas}

\subsection{Métricas orientadas a negocio}

Las métricas tradicionales de machine learning no capturan completamente el valor empresarial en aplicaciones energéticas. Desarrollamos métricas específicas que evalúan la utilidad práctica de las predicciones:

\subsubsection{Business Impact Score (BIS)}

\begin{equation}
BIS = w_1 \cdot \text{Billing Accuracy} + w_2 \cdot \text{Peak Prediction} + w_3 \cdot \text{Efficiency Optimization}
\end{equation}

Donde:
\begin{itemize}
\item Billing Accuracy evalúa precisión para facturación energética
\item Peak Prediction mide capacidad de predecir picos de demanda
\item Efficiency Optimization cuantifica potencial de ahorro energético
\end{itemize}

\subsubsection{Energy-Weighted Mean Absolute Error (EWMAE)}

Para aplicaciones energéticas, errores en períodos de alto consumo tienen mayor impacto económico:

\begin{equation}
EWMAE = \frac{1}{n} \sum_{i=1}^{n} w_i \cdot |y_i - \hat{y}_i|
\end{equation}

Donde $w_i = \frac{y_i}{\max(y)} + \epsilon$ pondera errores por nivel de consumo.

\section{Benchmarking contra Estado del Arte}

\subsection{Comparación con modelos de referencia}

La evaluación incluye comparación sistemática contra modelos de referencia establecidos en la literatura:

\begin{table}[H]
\centering
\caption{Comparación de rendimiento contra estado del arte}
\begin{tabular}{lrrrrr}
\toprule
\textbf{Modelo} & \textbf{MAE (W)} & \textbf{RMSE (W)} & \textbf{R²} & \textbf{MAPE (\%)} & \textbf{BIS} \\
\midrule
Persistence Baseline & 187.3 & 246.8 & 0.721 & 24.6 & 0.42 \\
ARIMA & 156.2 & 198.4 & 0.812 & 19.3 & 0.58 \\
Random Forest & 134.7 & 171.9 & 0.857 & 16.1 & 0.67 \\
LSTM & 121.3 & 158.2 & 0.879 & 14.8 & 0.72 \\
\textbf{Ensemble Propuesto} & \textbf{108.9} & \textbf{142.1} & \textbf{0.896} & \textbf{12.4} & \textbf{0.78} \\
\bottomrule
\end{tabular}
\label{tab:benchmark_comparison}
\end{table}

Los resultados demuestran que el ensemble propuesto supera consistentemente a los modelos de referencia en todas las métricas evaluadas, con mejoras particulares en precisión global (MAE 42\% mejor que baseline) y utilidad práctica (BIS 85\% superior).

\subsection{Validación cruzada temporal rigurosa}

La validación cruzada específica para series temporales energéticas respeta la estructura temporal y estacional de los datos mediante:

\begin{itemize}
    \item \textbf{Splits temporales con gaps:} Evita data leakage manteniendo períodos de separación entre entrenamiento y test
    \item \textbf{Respeto de estacionalidad:} Considera patrones diarios, semanales y estacionales
    \item \textbf{Validación múltiple:} Evaluación con múltiples métricas (MAE, RMSE, R², MAPE)
    \item \textbf{Análisis de estabilidad:} Verifica consistencia del rendimiento across different time periods
\end{itemize}

La implementación utiliza la clase \texttt{EnergyTimeSeriesCV} que genera splits temporales con períodos de test de 30 días y gaps de 7 días para garantizar independencia temporal.

Esta metodología integral de evaluación asegura que los modelos de predicción energética sean evaluados con rigor científico y proporciona métricas directamente aplicables a contextos empresariales y de investigación.
